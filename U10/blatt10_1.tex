\documentclass{article}
\usepackage[a4paper,left=10mm,right=10mm,top=15mm,bottom=15mm]{geometry}
\usepackage{amssymb,amsthm,latexsym,amsfonts, amsmath, bm}
\usepackage[lined,ruled]{algorithm2e}
\usepackage{extarrows}
\usepackage{enumerate}
\usepackage{tikz}
\usepackage{underoverlap}
\usepackage{colortbl}
\usepackage{graphicx}
\usepackage{float}
\usepackage{array}
\newtheorem*{lemma}{Lemma}
\newtheorem{theorem}{Theorem}
\title{Exercises for Algorithmic Bioinformatics II\\
Assignment 10}
\author{Xiheng He}
\date{Januar 2022}
\linespread{2.0}
\begin{document}
\flushright{Wintersemester 2021/22}
\flushleft{Xiheng He}
\flushleft{Lisanne Friedrich}
{\let\newpage\relax\maketitle}
\begin{flushleft}
\textbf{Task 1: Simple Orthodox and Bayesian Statistics (10P):}
\newline
\textbf{Coin tossing}
\newline
You are modelling a set of fair and biased coins. Fair coins show head and tail with equal probability,
whereas a biased coin will show head with a probability of 7/8.
\begin{enumerate}[(a)]
    \item In order to find out, whether a given coin is fair, you throw it 10 times. How are the proba-
    bilities of each outcome for every coin (e.g. 5x head)?
    \newline \\
    \begin{tabular}{|m{10em}|m{10em}|m{15em}|}
    \hline
    \textbf{outcome} & \textbf{fair coin} & \textbf{biased coin} \\
    \hline
    \textbf{0 head, 10 tail} & $\tbinom{10}{0}(\displaystyle\frac{1}{2})^{10} = 0.0010$ & $\tbinom{10}{0}(\displaystyle\frac{1}{8})^{10} = 0.0000$ \\ 
    \hline
    \textbf{1 head, 9 tail} & $\tbinom{10}{1}(\displaystyle\frac{1}{2})^{10} = 0.0098$ & $\tbinom{10}{1}(\displaystyle\frac{1}{8})^{9} \times \frac{7}{8} = 0.0000$ \\
    \hline
    \textbf{2 head, 8 tail} & $\tbinom{10}{2}(\displaystyle\frac{1}{2})^{10} = 0.0439$ & $\tbinom{10}{2}(\displaystyle\frac{1}{8})^{8} \times (\frac{7}{8})^2 = 0.0000$ \\
    \hline
    \textbf{3 head, 7 tail} & $\tbinom{10}{3}(\displaystyle\frac{1}{2})^{10} = 0.1172$ & $\tbinom{10}{3}(\displaystyle\frac{1}{8})^{7} \times (\frac{7}{8})^3 = 0.0000$ \\
    \hline
    \textbf{4 head, 6 tail} & $\tbinom{10}{4}(\displaystyle\frac{1}{2})^{10} = 0.1172$ & $\tbinom{10}{4}\tbinom{10}{4}(\displaystyle\frac{1}{8})^{6} \times (\frac{7}{8})^4 = 0.0005$ \\
    \hline
    \textbf{5 head, 5 tail} & $\tbinom{10}{5}(\displaystyle\frac{1}{2})^{10} = 0.2461$ & $\tbinom{10}{5}(\displaystyle\frac{1}{8})^{5} \times (\frac{7}{8})^5 = 0.0039$ \\
    \hline
    \textbf{6 head, 4 tail} & $\tbinom{10}{6}(\displaystyle\frac{1}{2})^{10} = 0.1172$ & $\tbinom{10}{6}(\displaystyle\frac{1}{8})^{4} \times (\frac{7}{8})^6 = 0.0230$ \\
    \hline
    \textbf{7 head, 3 tail} & $\tbinom{10}{7}(\displaystyle\frac{1}{2})^{10} = 0.1172$ & $\tbinom{10}{7}(\displaystyle\frac{1}{8})^{3} \times (\frac{7}{8})^7 = 0.0920$ \\
    \hline
    \textbf{8 head, 2 tail} & $\tbinom{10}{8}(\displaystyle\frac{1}{2})^{10} = 0.0439$ & $\tbinom{10}{8}(\displaystyle\frac{1}{8})^{2} \times (\frac{7}{8})^8 = 0.2416$ \\
    \hline
    \textbf{9 head, 1 tail} & $\tbinom{10}{9}(\displaystyle\frac{1}{2})^{10} = 0.0098$ & $\tbinom{10}{9}(\displaystyle\frac{1}{8})^{1} \times (\frac{7}{8})^9 = 0.3758$ \\
    \hline
    \textbf{10 head, 0 tail} & $\tbinom{10}{10}(\displaystyle\frac{1}{2})^{10} = 0.0010$ & $\tbinom{10}{10}(\displaystyle\frac{7}{8})^{10} = 0.2631$ \\  
    \hline
    \end{tabular}
    \item After how many tries would you be asserted that your coin is fair/not fair?
    \newline
    $X \sim binomial(10, \frac{1}{2})$ \\
    $H_0: Pr(head) = 0.5, H_1: Pr(head) \neq 0.5$ \\
    $p = Pr(x \leq 1) + Pr(x \geq 9) = 0.0010 + 0.0098 + 0.0098 + 0.0010 = 0.0216 < 0.05$ \\
    $p = Pr(x \leq 2) + Pr(x \geq 8) = 0.1094 > 0.05$ \\
    so we can only reject $H_0$ and assert coin is biased because of $p$-value < 0.05 significance when we tossed a coin 10 times and observed 0, 1, 9, 10 heads.
    \item Assume that you have 10 coins and that one of the coins is not fair. Would you adapt your
    trials from (b)?
    we use likelihood ratio to test whether the coin is fair or not.
    $H_0: Pr(head) = 0.5, H_1: Pr(head) \neq 0.5$ \\
    \begin{align*}
        \Lambda &= \frac{\mathcal{L} (\theta_0|x)}{\mathcal{L} (\theta_1|x)} \\
        &= \frac{P(d_1,\dots,d_N|H)}{P(d_1,\dots,d_N|-H)} \\
        & = \frac{9 \cdot \mathcal{L}(p = \frac{1}{2}|k, N)}{1 \cdot \mathcal{L}(p = \frac{7}{8}|k, N)} \\
        &= 9 \cdot (\frac{4}{7})^{N-k} \\
    \end{align*}
    $df = 10 \Longrightarrow \lambda_0 = x_{0.05}^2 = 18.31 > \Lambda \Longrightarrow $ So we Reject $H_0$.
    \newline
\textbf{Bayes}
\newline
Given a diagnostic method to determine a disease which appears in 1 of 10,000 in a population.
The method has a false positive rate of 0.001 and a false negative rate of 0.01.
    \item What is the probability of having the disease, if the test is positive?
    \newline
    A : Having disease B : Test positive
    \begin{align*}
        &P(B|\overline{A}) = 0.001, P(\overline{B}|A) = 0.01 \Longrightarrow P(\overline{B}|\overline{A}) = 0.999, P(B|A) = 0.99 \\
        &P(A) = \frac{1}{10000} = 0.0001 \Longrightarrow P(\overline{A}) = 0.9999 \\
        &P(B) = \sum_n P(B|A_n) = P(B|A)P(A) + P(B|\overline{A})P(\overline{A}) = 1.0989 \times 10^{-3} \Longrightarrow \\
        &P(A|B) = \frac{P(B|A)P(A)}{P(B)} = \frac{0.99 \cdot 0.0001}{1.0989 \times 10^{-3}} = 0.09009 \\
    \end{align*}
    The probability of having the disease when test is positive is 0.09009.
    \item What is the probability of not having the disease, if the test is negative?
    \newline \\
    \begin{align*}
        P(\overline{A}|\overline{B}) = \frac{P(\overline{B}|\overline{A})P(\overline{A})}{P(\overline{B})} = \frac{0.999 \cdot 0.9999}{1 - 1.0989 \times 10{-3}} = 0.9999
    \end{align*}
    The probability of not having the disease when test is negative is 0.9999.
\end{enumerate}
\end{flushleft}
\end{document}