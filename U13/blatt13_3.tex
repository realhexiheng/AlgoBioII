\documentclass[12pt]{article}
\usepackage[a4paper,left=10mm,right=10mm,top=15mm,bottom=15mm]{geometry}
\usepackage{amssymb,amsthm,latexsym,amsfonts, amsmath, bm}
\usepackage[lined,ruled]{algorithm2e}
\usepackage{extarrows}
\usepackage{enumerate}
\usepackage{tikz}
\usepackage{underoverlap}
\usepackage{colortbl}
\usepackage{graphicx}
\usepackage{float}
\usepackage{array}
\newtheorem*{lemma}{Lemma}
\newtheorem{theorem}{Theorem}
\title{Exercises for Algorithmic Bioinformatics II\\
Assignment 13}
\author{Xiheng He}
\date{Februar 2022}
\linespread{2.0}
\begin{document}
\flushright{Wintersemester 2021/22}
\flushleft{Xiheng He}
\flushleft{Lisanne Friedrich}
{\let\newpage\relax\maketitle}
\begin{flushleft}
\textbf{Exercise 3 (Markov-Chains II, 10P):}
\newline
A woman has three umbrellas, which are either in her office or her flat. Let $X_n$ be the number of
umbrellas at the place where she currently is. If she leaves her flat in the morning, or the office
in the afternoon, and it is currently raining, she will take an umbrella with her, if there is one.
Otherwise she has to take the bus. Consider that the probability of rain for each of her walks is 0.2
(independent of the previous walk).
\begin{enumerate}[(a)]
    \item Create a Markov chain for $X_n$ and determine the respective transition probabilities.
    \begin{itemize}
        \item The set of all states $S: = \{0, 1, 2, 3\}$
        \item The probabilities from state $x_n = 0$ to state $x_n = \{0, 1, 2\}$ are 0, 0, 0,
        as there are 3 umbrellas in total which meas $x_n + x_{n + 1} \geq 3$.
        Thus the probability from state $x_n = 0$ to state $x_n = 3$ is 1.
        Obviously, $P(x_n = 1, x_{n + 1} = 0) = P(x_n = 1, x_{n + 1} = 1) = P(x_n = 2, x_{n + 1} = 0) = 0$.
        \item $P(x_n = 2, x_{n + 1} = 3) = P(x_n = 3, x_{n + 1} = 2) = P(x_n = 3, x_{n + 1} = 3) = 0$ as
        the woman can only take one umbrella with her which means $x_n + x_{n + 1} \leq 4$.
        \item $P(x_n = 1, x_{n + 1} = 2) = P(x_n = 2, x_{n + 1} = 1) = P(x_n = 3, x_{n + 1} = 0) = 0.8$,
        as these imply that the woman has not taken a umbrella on a no-rainy day with $P = 1 - 0.2 = 0.8$.
        \item $P(x_n = 1, x_{n + 1} = 3) = P(x_n = 2, x_{n + 1} = 2) = P(x_n = 3, x_{n + 1} = 1) = 0.2$,
        as these imply that the woman took a umbrella on a rainy day with $P = 0.2$.
    \end{itemize}
    Thus, the state transition matrix $A$ is defined as: 
    \begin{equation*}
        A = 
        \left(\begin{array}{ccccc}
            x_n & 0 & 1 & 2 & 3 \\
            0 & 0 & 0 & 0 & 1 \\
            1 & 0 & 0 & 0.8 & 0.2 \\
            2 & 0 & 0.8 & 0.2 & 0 \\
            3 & 0.8 & 0.2 & 0 & 0
        \end{array}\right)
    \end{equation*}
    \item Calculate the stationary distribution of the Markov chain and derive from that, with which
    probability the woman has to take the bus.
    \newline
    Let $\pi = (a, b, c, d)$ for stationary distribution where $a, b, c, d$ refer to the probability of $x = \{0, 1, 2, 3\}$.
    \newline
    \begin{equation*}
        (a, b, c, d) = (a, b, c, d) \cdot
        \left(\begin{array}{cccc}
            0 & 0 & 0 & 1 \\
            0 & 0 & 0.8 & 0.2 \\
            0 & 0.8 & 0.2 & 0 \\
            0.8 & 0.2 & 0 & 0
        \end{array}\right)
    \end{equation*}
    \begin{align}
        & a \cdot 0 + b \cdot 0 + c \cdot 0 + d \cdot 0.8 = 0.8d = a \\
        & a \cdot 0 + b \cdot 0 + c \cdot 0.8 + d \cdot 0.2 = 0.8c + 0.2d = b \\
        & a \cdot 0 + b \cdot 0.8 + c \cdot 0.2 + d \cdot 0 = 0.8b + 0.2c = c \\
        & a \cdot 1 + b \cdot 0.2 + c \cdot 0 + d \cdot 0 = a + 0.2b = d
    \end{align}
    \newpage
    \begin{itemize}
        \item from (3): $b = c$
        \item from (2): $b = c = d$
        \item thus, $a : b : c : d = 4 : 5 : 5 : 5$, $\pi = (\displaystyle\frac{4}{19}, \displaystyle\frac{5}{19}, \displaystyle\frac{5}{19}, \displaystyle\frac{5}{19})$
        \item therefore, the probability of taking bus (there is no umbrella AND it is rainy outside) is $P(x = 0 | rain) = \displaystyle\frac{4}{19} \times 0.2 \approx 0.042$
    \end{itemize}
\end{enumerate}
\end{flushleft}
\end{document}