\documentclass{article}
\usepackage[a4paper,left=15mm,right=15mm,top=15mm,bottom=15mm]{geometry}
\usepackage{amssymb,amsthm,latexsym,amsfonts, amsmath, bm}
\usepackage[lined,ruled]{algorithm2e}
\usepackage{extarrows}
\usepackage{enumerate}
\usepackage{tikz}
\usepackage{underoverlap}
\usepackage{xcolor}
\usepackage{graphicx}
\usepackage{float}
\newtheorem*{lemma}{Lemma}
\newtheorem{theorem}{Theorem}
\title{Exercises for Algorithmic Bioinformatics II\\
Assignment 1}
\author{Xiheng He}
\date{November 2021}
\linespread{1.6}
\begin{document}
\flushright{Wintersemester 2021/22}
\flushleft{Xiheng He}
\flushleft{Lisanne Friedrich}
{\let\newpage\relax\maketitle}
\begin{flushleft}
\textbf{Exercise 1 (Chang-Lawler proof, 10P):}
\newline
Prove the following lemma (also given in the lecture slides):
\begin{lemma}
    \normalfont
    When the CL search excludes a region $R$ of $T$, then there is no $k$-approximate match of
    $P$ in $T$ that completely contains region $R$.
\end{lemma}
Is it also guaranteed that a region is excluded if it contains $\geq  k + 1$ mismatches?
Prove it or give a counter-example.
\newline \\
When count $> k$ loop for searching terminated too early. That leads to more than $k$ mismatches in the region $R$,
which means that the region should be excluded. That contradicts the lemma.
\newline \\
counter-example:
\newline
pattern: baaaccaaaabacaba
\newline
k = 2
\newline
text: aaaabaaacacbabbbabaaaccaaaabacabacccbcaaabcacb
\newline
region included: acccbcaa
\newline
mismatch = 3
\end{flushleft}
\end{document}